\documentclass[a4paper, 10pt]{article}
\hyphenpenalty=8000
\textwidth=125mm
\textheight=185mm

\usepackage{graphicx}
\usepackage{alltt}
\usepackage{amsmath}
\usepackage[hidelinks, pdftex]{hyperref}

\pagenumbering{arabic}
\setcounter{page}{1}
\renewcommand{\thefootnote}{\fnsymbol{footnote}}

\begin{document}

\begin{center}
\LARGE
\textbf{Documentación del Código de Simulación de Escena}\[]
\small
\textbf{ROBBYEL ELIAS, CARLOS JERONIMO, ANDRUS LOPEZ}\[]
\end{center}

\begin{abstract}
Este documento describe la implementación de una simulación de escena en Python utilizando clases y animaciones. Se explican los métodos principales y sus funcionalidades, incluyendo la generación de escenas con rectángulos y puntos, su modificación, desplazamiento y animación.
\end{abstract}

\section{Introducción}
Este documento presenta la documentación detallada del código de simulación de escena. La simulación consiste en la representación gráfica de un área rectangular con puntos aleatorios y su animación mediante desplazamientos en distintas direcciones.

\section{Descripción del Código}

El código está estructurado en una clase llamada \texttt{Escena}, que encapsula todas las funcionalidades necesarias para manejar y modificar una representación gráfica.

\subsection{Inicialización de la Escena}
El constructor de la clase \texttt{Escena} inicializa los parámetros básicos:
\begin{itemize}
    \item \texttt{ancho}, \texttt{alto}: dimensiones del área rectangular.
    \item \texttt{color}: color del rectángulo de fondo.
    \item \texttt{n\_puntos}: cantidad de puntos generados aleatoriamente.
\end{itemize}

\subsection{Método para Dibujar la Escena}
El método \texttt{dibujar\_escena} representa gráficamente el rectángulo y los puntos generados. Se usa la librería \texttt{matplotlib} para trazar el rectángulo y los puntos dentro de la escena.

\subsection{Desplazamiento de Puntos}
El método \texttt{desplazar\_puntos(valor, direccion)} permite mover los puntos en diferentes direcciones:\\
\textbf{Parámetros:}
\begin{itemize}
    \item \texttt{valor}: cantidad de desplazamiento.
    \item \texttt{direccion}: dirección del movimiento (\texttt{'derecha'}, \texttt{'izquierda'}, \texttt{'arriba'}, \texttt{'abajo'}).
\end{itemize}

\subsection{Animación de la Escena}
El método \texttt{animar(velocidad, direccion)} genera una animación de los puntos desplazándose en la dirección especificada. Se usa la función \texttt{FuncAnimation} de \texttt{matplotlib.animation}.

\section{Código Fuente}

\begin{alltt}
import numpy as np
import matplotlib.pyplot as plt
import matplotlib.patches as patches
import matplotlib.animation as animation
from IPython.display import HTML, display

class Escena:
    def _init_(self, ancho, alto, color, n_puntos):
        self.ancho = ancho
        self.alto = alto
        self.color = color
        self.puntos = np.random.rand(n_puntos, 2) * [ancho, alto]

    def cambiar_escena(self, ancho, alto, color, n_puntos):
        self._init_(ancho, alto, color, n_puntos)

    def dibujar_escena(self):
        fig, ax = plt.subplots()
        ax.set_xlim(0, self.ancho)
        ax.set_ylim(0, self.alto)
        rect = patches.Rectangle((0, 0), self.ancho, self.alto, color=self.color)
        ax.add_patch(rect)
        ax.scatter(self.puntos[:, 0], self.puntos[:, 1], color='red', label='Puntos')
        ax.legend()
        plt.show()

    def desplazar_puntos(self, valor, direccion):
        desplazamientos = {
            'derecha': np.array([valor, 0]),
            'izquierda': np.array([-valor, 0]),
            'arriba': np.array([0, valor]),
            'abajo': np.array([0, -valor])
        }
        if direccion in desplazamientos:
            self.puntos += desplazamientos[direccion]

    def animar(self, velocidad, direccion):
        fig, ax = plt.subplots()
        ax.set_xlim(0, self.ancho)
        ax.set_ylim(0, self.alto)
        rect = patches.Rectangle((0, 0), self.ancho, self.alto, color=self.color)
        ax.add_patch(rect)
        scatter = ax.scatter(self.puntos[:, 0], self.puntos[:, 1], color='red')

        desplazamiento = {
            'derecha': np.array([velocidad, 0]),
            'izquierda': np.array([-velocidad, 0]),
            'arriba': np.array([0, velocidad]),
            'abajo': np.array([0, -velocidad])
        }.get(direccion, np.array([0, 0]))

        def update(frame):
            self.puntos[:] += desplazamiento
            scatter.set_offsets(self.puntos)
            return scatter,

        ani = animation.FuncAnimation(fig, update, frames=20, interval=100, blit=True)
        return HTML(ani.to_jshtml())

if _name_ == "_main_":
    escena = Escena(10, 5, 'blue', 10)
    escena.dibujar_escena()
    escena.desplazar_puntos(1, 'derecha')
    escena.dibujar_escena()
    display(escena.animar(0.5, 'arriba'))
\end{alltt}

\section{Conclusiones}
Se presentó un código en Python para simular una escena gráfica con puntos en movimiento. Se explicaron las funciones principales y su implementación, proporcionando una visión clara del proceso de desarrollo.

\end{document}